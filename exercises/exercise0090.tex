\begin{opgave}
Het ruimteschip "Raket naar Jupiter" moet asteroïden ontwijken tijdens zijn 
reis. De computer telt automatisch: in de eerste 2 uur ontwijken ze 18 
asteroïden per uur, in de volgende 3 uur ontwijken ze 24 asteroïden per uur, 
en in de laatste 4 uur ontwijken ze 15 asteroïden per uur. Hoeveel asteroïden 
hebben ze in totaal ontwijken tijdens de 9-urige reis?
\end{opgave}

\begin{oplossing}
Asteroïden per periode:
Eerste 2 uur: 2 x 18 = 36 asteroïden
Volgende 3 uur: 3 x 24 = 72 asteroïden
Laatste 4 uur: 4 x 15 = 60 asteroïden

Totaal ontwijken asteroïden:
36 + 72 + 60 = 168 asteroïden

Antwoord: 168 asteroïden
\end{oplossing}