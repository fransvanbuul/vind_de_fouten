\begin{opgave}
Je vergelijkt de prijzen van verschillende saxofoons bij 3 muziekwinkels:

\begin{itemize}
\item Winkel A: Alt saxofoon \texteuro{1.250}, Tenor saxofoon \texteuro{1.890}
\item Winkel B: Alt saxofoon \texteuro{1.180}, Tenor saxofoon \texteuro{1.950}
\item Winkel C: Alt saxofoon \texteuro{1.320}, Tenor saxofoon \texteuro{1.820}
\end{itemize}

Je wilt beide saxofoons kopen bij dezelfde winkel voor korting. Winkel A geeft
8\% korting op de totaalprijs, Winkel B geeft 5\% korting, Winkel C geeft 10\%
korting. Bij welke winkel betaal je het minst en hoeveel?
\end{opgave}

\begin{oplossing}
Totaalprijs per winkel voor korting:
Winkel A: \texteuro{1.250} + \texteuro{1.890} = \texteuro{3.140}
Winkel B: \texteuro{1.180} + \texteuro{1.950} = \texteuro{3.130}
Winkel C: \texteuro{1.320} + \texteuro{1.820} = \texteuro{3.140}

Na korting:
Winkel A: \texteuro{3.140} - (8\% van \texteuro{3.140}) = \texteuro{3.140} - \texteuro{251,20} = \texteuro{2.888,80}
Winkel B: \texteuro{3.130} - (5\% van \texteuro{3.130}) = \texteuro{3.130} - \texteuro{156,50} = \texteuro{2.973,50}
Winkel C: \texteuro{3.140} - (10\% van \texteuro{3.140}) = \texteuro{3.140} - \texteuro{314,00} = \texteuro{2.826,00}

Antwoord: Winkel C is goedkoopst met \texteuro{2.826}
\end{oplossing}