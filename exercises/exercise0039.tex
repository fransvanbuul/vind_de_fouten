\begin{opgave}
Voor je raket experiment heb je verschillende motoren getest. De hoogte die
elke motor bereikt volgt een bepaald patroon:

\begin{itemize}
\item Motor A: 15, 22, 29, 36, 43 meter (5 tests)
\item Motor B: 8, 16, 32, 64, 128 meter (5 tests)
\item Motor C: 25, 23, 26, 24, 27 meter (5 tests)
\end{itemize}

Je wilt nog 2 tests doen met elke motor. Wat verwacht je dat de resultaten
worden van test 6 en 7 voor elke motor? Welke motor bereikt bij test 7 de
hoogste verwachte hoogte?
\end{opgave}

\begin{oplossing}
Motor A patroon: +7 elke test (15, 22, 29, 36, 43)
Test 6: 43 + 7 = 50 meter
Test 7: 50 + 7 = 57 meter

Motor B patroon: x2 elke test (8, 16, 32, 64, 128)
Test 6: 128 x 2 = 256 meter
Test 7: 256 x 2 = 512 meter

Motor C patroon: +2, -2, +2, -2 wisselend (25, 23, 26, 24, 27)
Test 6: 27 - 2 = 25 meter (want -2 na +2)
Test 7: 25 + 2 = 27 meter (want +2 na -2)

Hoogste bij test 7: Motor B met 512 meter

Antwoord: Motor B bereikt hoogste hoogte (512 meter)
\end{oplossing}