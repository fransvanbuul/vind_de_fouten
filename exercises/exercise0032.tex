\begin{opgave}
Harry gaat winkelen in de Wegisweg en krijgt van verschillende winkels korting:

\begin{itemize}
\item Ollivanders (toverstokken): 15\% korting op 280 galjoenen
\item Florissa \& Flodders (schoolboeken): 20\% korting op 150 galjoenen  
\item Goudgrijps bank (wisselkoers): 10\% toeslag op 95 galjoenen
\item Madam Malleins (mantels): 25\% korting op 120 galjoenen
\end{itemize}

Harry heeft 500 galjoenen. Kan hij alles kopen? Hoeveel galjoenen heeft hij
over of tekort na alle aankopen?
\end{opgave}

\begin{oplossing}
Ollivanders: 15\% korting op 280 = 280 - (280 x 15\%) = 280 - 42 = 238 galjoenen
Florissa \& Flodders: 20\% korting op 150 = 150 - (150 x 20\%) = 150 - 30 = 120 galjoenen
Goudgrijps: 10\% toeslag op 95 = 95 + (95 x 10\%) = 95 + 9,50 = 104,50 galjoenen
Madam Malleins: 25\% korting op 120 = 120 - (120 x 25\%) = 120 - 30 = 90 galjoenen

Totaal: 238 + 120 + 104,50 + 90 = 552,50 galjoenen
Beschikbaar: 500 galjoenen
Te kort: 552,50 - 500 = 52,50 galjoenen

Antwoord: 52,50 galjoenen te kort
\end{oplossing}