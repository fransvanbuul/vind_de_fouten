\begin{opgave}
Voor het scheikundeproject ga je veilige experimenten doen met bruistabletten. Je hebt 144
bruistabletten gekocht. Voor elk experiment gebruik je 6 tabletten.

\begin{itemize}
\item Experiment 1: Kleur verandering (herhaal 8 keer)
\item Experiment 2: Volume meting (herhaal 5 keer)  
\item Experiment 3: Temperatuur onderzoek (herhaal 7 keer)
\end{itemize}

Hoeveel tabletten gebruik je in totaal? Hoeveel tabletten houd je over voor extra
experimenten?
\end{opgave}

\begin{oplossing}
Experiment 1: 8 x 6 = 48 tabletten.
Experiment 2: 5 x 6 = 30 tabletten.
Experiment 3: 7 x 6 = 42 tabletten.
Totaal gebruikt: 48 + 30 + 42 = 120 tabletten.
Over: 144 - 120 = 24 tabletten.
\end{oplossing}