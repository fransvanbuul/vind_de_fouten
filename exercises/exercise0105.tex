\begin{opgave}

\begin{tekstmetfouten}
Afgelopen zaterdag heb ik samen met mijn moeder koekjes gebakken.\\
We hadden besloten om chocoladekoekjes te maken omdat die mijn\\
favoriet zijn. Eerst moeste we alle ingredienten verzamelen. We\\
hadden bloem, suiker, boter, eieren en chocolade stukjes nodig.\\
Mijn moeder haalde een grote kom uit de kast en gaf mij een\\
houten lepel. Ik mocht beginnen met het mengen van de boter en\\
de suiker. Dat was best zwaar werk want de boter was nog koud\\
en hard. Na een tijdje werd het mengsel zachter en begon het er\\
goed uit te zien. Toen voegde ik de eieren toe en roerde alles\\
goed door elkaar. Mijn moeder zeefde ondertussen de bloem. Ze\\
zei dat je dat altijd moet doen om klontjes te voorkomen. Daarna\\
voegden we de bloem langzaam toe aan het beslag terwijl ik bleef\\
roeren. Het laatste wat we toevoegden waren de chocolade stukjes.\\
Daar was ik het meest blij mee want ik ben gek op chocolade. We\\
verwarmden de oven voor op 180 graden. Mijn moeder pakte een\\
bakplaat en belegde die met bakpapier. Met een lepel maakte we\\
kleine hoopjes beslag op de plaat. Je moet er wel op letten dat\\
je genoeg ruimte tussen de koekjes laat want ze worden groter in\\
de oven. We zette de plaat in de oven en stellen de timer in op\\
twaalf minuten. Terwijl de koekjes bakten, ruimden we de keuken\\
op. Na twaalf minuten haalde mijn moeder de bakplaat uit de oven\\
met een ovenhandschoen. De koekjes zagen er heerlijk uit en\\
roken zalig. We moesten wel even wachten voordat we ze konden\\
proeven omdat ze nog te heet waren. Uiteindelijk mochten we er\\
allebei drie opeten. De rest bewaarden we in een trommel.
\end{tekstmetfouten}

\begin{vragen}
\vraag{3}{Eerst moeste we alle ingredienten verzamelen.}%
{moeste}%
{moeste (zoals in de tekst)}%
{moesten}%
{moest}%
{moeste we}

\vraag{4}{We hadden bloem ... en chocolade stukjes nodig.}%
{chocolade stukjes}%
{chocolade stukjes (zoals in de tekst)}%
{chocoladestukjes}%
{chocolade-stukjes}%
{chocalade stukjes}

\vraag{16}{Met een lepel maakte we kleine hoopjes beslag op de plaat.}%
{maakte we}%
{maakte we (zoals in de tekst)}%
{maakten we}%
{maken we}%
{maakte wij}

\vraag{19}{We zette de plaat ... en stellen de timer in op twaalf minuten.}%
{stellen}%
{stellen (zoals in de tekst)}%
{stelde}%
{stelden}%
{stelt}

\vraag{22}{De bakplaat uit de oven met een ovenhandschoen.}%
{ovenhandschoen}%
{ovenhandschoen (zoals in de tekst)}%
{oven handschoen}%
{oven-handschoen}%
{ovenhand schoen}

\end{vragen}

\end{opgave}

\begin{oplossing}
\begin{enumerate}
\item C - moesten is correct (meervoud: wij moesten)
\item C - chocoladestukjes is de juiste spelling (samenstelling, geen
spatie)
\item C - maakten we is correct (meervoud verleden tijd: wij maakten)
\item C - stelden is correct (verleden tijd om consistent te blijven met
zette)
\item A - ovenhandschoen is correct (samenstelling zonder spatie, zoals in
de tekst)
\end{enumerate}
\end{oplossing}
