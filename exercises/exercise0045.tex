\begin{opgave}
Je wilt lid worden van de tennisclub. De kosten zijn:

\begin{itemize}
\item Lidmaatschap per jaar: \texteuro{240}
\item Particuliere lessen: \texteuro{35} per les
\item Groepslessen: \texteuro{18} per les
\item Racket huren per maand: \texteuro{15}
\end{itemize}

Je plant voor het eerste jaar: 12 particuliere lessen, 20 groepslessen,
en je huurt 8 maanden een racket. Je ouders betalen 60\% van alle kosten.
Hoeveel moet jij zelf betalen?
\end{opgave}

\begin{oplossing}
Totale kosten:
Lidmaatschap: \texteuro{240}
Particuliere lessen: 12 x \texteuro{35} = \texteuro{420}
Groepslessen: 20 x \texteuro{18} = \texteuro{360}
Racket huren: 8 x \texteuro{15} = \texteuro{120}
Totaal: \texteuro{240} + \texteuro{420} + \texteuro{360} + \texteuro{120} = \texteuro{1.140}

Ouders betalen: 60\% van \texteuro{1.140} = \texteuro{1.140} x 0,60 = \texteuro{684}
Jij betaalt: \texteuro{1.140} - \texteuro{684} = \texteuro{456}

Antwoord: \texteuro{456}
\end{oplossing}