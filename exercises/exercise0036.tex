\begin{opgave}
Voor het survivalkamp van de scouting moet je een waterfiltratiesysteem maken.
Je hebt verschillende materialen nodig die in verschillende eenheden verkocht worden:

\begin{itemize}
\item Activated carbon: 2,5 kg nodig, verkocht per 500 gram voor \texteuro{8,50}
\item Zand: 12 kg nodig, verkocht per 3 kg zakken voor \texteuro{4,75}
\item Grind: 8.000 gram nodig, verkocht per 2,5 kg voor \texteuro{6,25}
\item PVC buis: 150 cm nodig, verkocht per 75 cm voor \texteuro{12,30}
\end{itemize}

Hoeveel eenheden moet je van elk materiaal kopen? Wat zijn de totale kosten?
\end{opgave}

\begin{oplossing}
Activated carbon: 2,5 kg = 2.500 gram
Nodig: 2.500 \textrm{\textdiv} 500 = 5 pakken van 500 gram
Kosten: 5 x \texteuro{8,50} = \texteuro{42,50}

Zand: 12 kg nodig
Nodig: 12 \textrm{\textdiv} 3 = 4 zakken van 3 kg
Kosten: 4 x \texteuro{4,75} = \texteuro{19,00}

Grind: 8.000 gram = 8 kg
Nodig: 8 \textrm{\textdiv} 2,5 = 3,2 → 4 zakken van 2,5 kg (kan niet delen)
Kosten: 4 x \texteuro{6,25} = \texteuro{25,00}

PVC buis: 150 cm nodig
Nodig: 150 \textrm{\textdiv} 75 = 2 stukken van 75 cm
Kosten: 2 x \texteuro{12,30} = \texteuro{24,60}

Totaal: \texteuro{42,50} + \texteuro{19,00} + \texteuro{25,00} + \texteuro{24,60} = \texteuro{111,10}

Antwoord: \texteuro{111,10}
\end{oplossing}