\begin{opgave}
Tijdens een Monopoly bank toernooi moet Sofia precies bijhouden hoeveel geld 
elke speler heeft. Speler A begint met 1500 euro, koopt 2 straten voor 180 
euro per stuk, en ontvangt 240 euro huur. Speler B begint met 1500 euro, 
koopt 1 hotel voor 400 euro, en betaalt 150 euro belasting. Hoeveel geld 
hebben beide spelers samen aan het einde?
\end{opgave}

\begin{oplossing}
Speler A berekenen:
Begint met: 1500 euro
Koopt straten: 2 x 180 = 360 euro
Na aankoop: 1500 - 360 = 1140 euro
Ontvangt huur: 1140 + 240 = 1380 euro

Speler B berekenen:
Begint met: 1500 euro
Koopt hotel: 1500 - 400 = 1100 euro
Betaalt belasting: 1100 - 150 = 950 euro

Totaal samen:
1380 + 950 = 2330 euro

Antwoord: 2330 euro
\end{oplossing}