\begin{opgave}

\begin{tekstmetfouten}
Gisteren had ik een geweldig avontuur in Minecraft. Ik vondt\\
een verlaten mijnschacht diep onder de grond. Het was al laat\\
in de middag toen ik besloot om naar beneden te gaan. Mijn\\
rugzak zat vol met fakkels en eten voor onderweg. In de schachten\\
lagen overal rails en oude karretjes. Sommige karretjes waren\\
nog gevuld met kolen en ijzererts. Het leek erop dat iemand\\
hier lang geleden hard had gewerkt. Mijn vriend Tim zei dat\\
we voorzichtig moesten zyn, want er kunnen monsters verschijnen\\
in het donker. We hadde allebei een zwaard en een schild bij\\
ons voor de zekerheid. Ik plaatste overal fakkels om de weg\\
te verlichten. Na een tijdje ontdekten we een grote grot met\\
glinsterende diamanten aan het plafond en de wanden! Het was\\
een prachtig gezicht. We waren zo blij dat we meteen begonnen\\
met graven. We moesten voorzichtig zijn om niet in de lava te\\
vallen die we beneden in de grot konden zien. Tim heeft een\\
speciale pikhouweel met betovering waarmee hij sneller kan\\
graven. Hij verzamelde de diamanten terwijl ik de wacht hield.\\
Na een uur hadden we genoeg diamanten voor nieuwe zwaarden en\\
een harnas. We klommen terug naar boven met onze schatten.\\
Buiten was het ondertussen donker geworden. Tim vind diamanten\\
het mooiste materiaal in het spel. Ik ben het helemaal met hem\\
eens. Morgen gaan we verder met het bouwen van ons kasteel.
\end{tekstmetfouten}

\begin{vragen}
\vraag{1}{Ik vondt een verlaten mijnschacht ... de grond.}%
{vondt}%
{vondt (zoals in de tekst)}%
{vond}%
{vonden}%
{vinden}

\vraag{8}{In de schachten lagen ... oude karretjes.}%
{karretjes}%
{karretjes (zoals in de tekst)}%
{karetjes}%
{karrretjes}%
{karretjus}

\vraag{11}{Tim zei dat we ... moesten zyn ... in het donker.}%
{zyn}%
{zyn (zoals in de tekst)}%
{zijn}%
{zein}%
{zeyn}

\vraag{12}{We hadde allebei ... bij ons voor de zekerheid.}%
{hadde}%
{hadde (zoals in de tekst)}%
{hadden}%
{hade}%
{heeft}

\vraag{23}{Tim vind diamanten het mooiste materiaal in het spel.}%
{vind}%
{vind (zoals in de tekst)}%
{vinden}%
{vindt}%
{vonden}
\end{vragen}

\end{opgave}

\begin{oplossing}
\begin{enumerate}
\item B - vond is de juiste verleden tijd van vinden (ik vondt bestaat niet)
\item A - karretjes is de correcte spelling (zoals in de tekst)
\item B - zijn is de juiste spelling (geen y in het Nederlands voor dit woord)
\item B - hadden is correct (meervoud: wij hadden)
\item C - vindt is correct (tegenwoordige tijd, derde persoon: Tim vindt)
\end{enumerate}
\end{oplossing}
