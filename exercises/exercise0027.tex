\begin{opgave}
Je bakt koekjes om te verkopen op de schoolmarkt. De ingrediënten kosten:

\begin{itemize}
\item Bloem (1 kg): \texteuro{2,50}
\item Boter (500g): \texteuro{4,25}
\item Suiker (1 kg): \texteuro{1,85}
\item Eieren (12 stuks): \texteuro{3,60}
\item Chocoladestukjes (300g): \texteuro{3,20}
\end{itemize}

Je koopt 3 kg bloem, 2 pakken boter, 2 kg suiker, 2 dozen eieren en 5 zakjes
chocoladestukjes. Je verkoopt alle koekjes voor \texteuro{45}. Hoeveel winst
maak je?
\end{opgave}

\begin{oplossing}
Kosten ingrediënten:
Bloem: 3 x \texteuro{2,50} = \texteuro{7,50}
Boter: 2 x \texteuro{4,25} = \texteuro{8,50}
Suiker: 2 x \texteuro{1,85} = \texteuro{3,70}
Eieren: 2 x \texteuro{3,60} = \texteuro{7,20}
Chocoladestukjes: 5 x \texteuro{3,20} = \texteuro{16,00}
Totale kosten: \texteuro{7,50} + \texteuro{8,50} + \texteuro{3,70} + \texteuro{7,20} + \texteuro{16,00} = \texteuro{42,90}

Winst: \texteuro{45,00} - \texteuro{42,90} = \texteuro{2,10}

Antwoord: \texteuro{2,10}
\end{oplossing}