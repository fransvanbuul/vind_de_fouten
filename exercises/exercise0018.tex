\begin{opgave}
Je speelt al 3 jaar saxofoon en je instrument heeft onderhoud nodig. De muziekwinkel rekent
de volgende kosten:

\begin{itemize}
\item Algemene reiniging en smering: 45 euro
\item 2 nieuwe rieten: 18 euro per stuk
\item Nieuwe mondstukdop: 25 euro
\item Extra controle \& afstelling: 35 euro
\end{itemize}

Je ouders geven je 150 euro voor het onderhoud. Is dit genoeg? Zo nee, hoeveel moet je er
zelf bijleggen? Zo ja, kun je dan ook nog nieuwe bladmuziek kopen voor 12 euro per boek?
\end{opgave}

\begin{oplossing}
Kosten onderhoud: 45 + (2 x 18) + 25 + 35 = 45 + 36 + 25 + 35 = 141 euro.
Van ouders: 150 euro, dus 150 - 141 = 9 euro over.
Bladmuziek kost 12 euro per boek, dus dat kan niet meer.
Antwoord: 9 euro over, geen bladmuziek
\end{oplossing}