\begin{opgave}
Je wilt keramiek maken met klei na de regenachtige dag. Je hebt 15 kilogram klei verzameld
uit de modderige tuin. Voor elk keramiek object heb je de volgende hoeveelheden klei nodig:

\begin{itemize}
\item Kleine kommen: 750 gram per stuk
\item Bloempotten: 1.200 gram per stuk
\item Beeldjes: 500 gram per stuk
\end{itemize}

Je wilt 6 kleine kommen, 4 bloempotten en 8 beeldjes maken. Heb je genoeg klei? Zo ja,
hoeveel gram klei houd je over? Zo nee, hoeveel gram te kort?
\end{opgave}

\begin{oplossing}
Benodigde klei:
Kommen: 6 x 750 = 4.500 gram
Bloempotten: 4 x 1.200 = 4.800 gram  
Beeldjes: 8 x 500 = 4.000 gram
Totaal: 4.500 + 4.800 + 4.000 = 13.300 gram
Beschikbaar: 15 kilogram = 15.000 gram
Over: 15.000 - 13.300 = 1.700 gram over
\end{oplossing}