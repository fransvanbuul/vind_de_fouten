\begin{opgave}

\begin{tekstmetfouten}
Afgelopen zaterdag had ik een belangrijke tennistoernooi. Ik was\\
een beetje zenuwachtig omdat ik tegen sterke tegenstanders moest\\
spelen. Mijn trainer zei dat ik goed had getraind en dat ik er\\
klaar voor was. Om acht uur 's ochtends begon de eerste wedstrijd.\\
Mijn tegenstander was een jongen uit Rotterdam die heel snel kon\\
serveren. De eerste set ging gelijk op. Het stond steeds gelijk\\
tot het 5-5 werd. Toen maakte ik een mooie passing shot en won\\
ik de set met 7-5. Iedereen klapte voor mij. Mijn ouders zaten\\
op de tribune en moedigden me aan. In de tweede set speelde ik\\
nog beter. Ik maakte weinig fouten en won met 6-3. Na de wedstrijd\\
schudde ik mijn tegenstander de hand. Hij zei dat ik goed had\\
gespeeld. In de tweede ronde moest ik tegen een meisje uit\\
Utrecht. Zij was de nummer twee van het toernooi. De wedstrijd\\
was heel spannend want we waren allebei even sterk. Het duurde\\
meer dan twee uur voordat er een winnaar was. Uiteindelijk won\\
zij met 7-6 in de derde set. Ik was teleurgesteld maar ook trots\\
op mezelf. Mijn trainer vind dat ik grote vooruitgang heb geboekt.\\
Hij denkt dat ik volgend jaar misschien kan winnen. Volgende week\\
ga ik weer extra hard trainen. Ik wil mijn backhand verbeteren\\
en sneller leren lopen. Op Dinsdag heb ik een speciale training\\
met een nieuwe coach. Die bal waarmee we trainen is zwaarder dan\\
een normale bal. Dat helpt om meer kracht te ontwikkelen. Mijn\\
doel is om volgend seizoen bij de beste tien spelers van mijn\\
leeftijdsgroep te komen. Daar werk ik hard voor.
\end{tekstmetfouten}

\begin{vragen}
\vraag{1}{Afgelopen zaterdag had ik een belangrijke tennistoernooi.}%
{belangrijke tennistoernooi}%
{belangrijke tennistoernooi (zoals in de tekst)}%
{belangrijke tennistoernooien}%
{belangrijk tennistoernooi}%
{belangrijks tennistoernooi}

\vraag{17}{Mijn trainer vind dat ik ... geboekt.}%
{vind}%
{vind (zoals in de tekst)}%
{vinden}%
{vindt}%
{vond}

\vraag{20}{Op Dinsdag heb ik een speciale training met een nieuwe coach.}%
{Op Dinsdag}%
{Op Dinsdag (zoals in de tekst)}%
{op Dinsdag}%
{op dinsdag}%
{Op dinsdag}

\vraag{21}{Die bal waarmee we trainen ... normale bal.}%
{Die bal}%
{Die bal (zoals in de tekst)}%
{Deze bal}%
{Dat bal}%
{Dit bal}

\vraag{14}{De wedstrijd was ... spannend want we waren ... even sterk.}%
{spannend want we}%
{spannend want we (zoals in de tekst)}%
{spannend, want we}%
{spannend Want we}%
{spannend; want we}
\end{vragen}

\end{opgave}

\begin{oplossing}
\begin{enumerate}
\item C - een belangrijk tennistoernooi (het-woord, dus belangrijk, niet
belangrijke)
\item C - vindt is correct (3e persoon enkelvoud: mijn trainer vindt)
\item C - op dinsdag (dagen van de week krijgen geen hoofdletter in het
Nederlands)
\item A - Die bal is correct (de-woord, ver weg/niet specifiek hier bedoeld)
\item B - spannend, want we (komma voor nevenschikkend voegwoord want)
\end{enumerate}
\end{oplossing}
