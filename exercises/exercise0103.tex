\begin{opgave}

\begin{tekstmetfouten}
Gisteravond speelden we met het hele gezin een spannend potje\\
Catan. Mijn vader had het spel voor mijn verjaardag gekocht en\\
we waren allemaal erg enthousiast. In het begin van het spel\\
kreeg iedereen twee dorpen en twee wegen. Ik koos een goede plek\\
bij de haven waar ik schaap kon ruilen. Mijn zus bouwdte haar\\
eerste dorp bij de bergen omdat ze veel erts wilde verzamelen.\\
Dat leek een slimme strategie. Na een paar beurten hadde we\\
allebei genoeg grondstoffen om uit te breiden. Ik bouwde een weg\\
naar een nieuwe plek met tarwe en hout. Deze grondstoffen zijn\\
heel belangrijk om meer wegen en dorpen te kunnen bouwen. Mijn\\
moeder had ondertussen al drie dorpen gebouwd. Ze was duidelijk\\
aan het winnen. Mijn vader probeerde de langste weg te maken\\
maar ik blokkeerde hem met een slim geplaatst dorp. Hij was niet\\
blij met deze actie. Toen gooi ik een zeven met de dobbelstenen.\\
Dat betekende dat de rover verplaatst moest worden. Ik zette hem\\
op het veld van mijn zus zodat zij geen erts meer kon krijgen.\\
Ze vond dit niet leuk en zei dat ik vals speelde. Maar het hoort\\
bij het spel om elkaar dwars te zitten. Na een uur spelen had\\
mijn moeder negen punten. Nog een punt en ze zou winnen. We\\
probeerden haar allemaal tegen te houden. Ik ruilde snel wat\\
grondstoffen en kocht een ontwikkelingskaart. Het was een\\
ridderkaar! Daarmee kon ik de grootste riddermacht overnemen\\
en kreeg ik twee extra punten. Plotseling had ik tien punten\\
en won ik het spel. Mijn familie was verbaasd over deze snelle\\
overwinning. Volgend keer moet ik beter op mijn punten letten.
\end{tekstmetfouten}

\begin{vragen}
\vraag{5-6}{Mijn zus bouwdte haar eerste dorp bij de bergen ... verzamelen.}%
{bouwdte}%
{bouwdte (zoals in de tekst)}%
{bouwde}%
{bouwte}%
{bouwt}

\vraag{7-8}{Na een paar ... hadde we allebei genoeg grondstoffen ... breiden.}%
{hadde}%
{hadde (zoals in de tekst)}%
{hadden}%
{had}%
{hebben}

\vraag{14}{Toen gooi ik een zeven met de dobbelstenen.}%
{gooi}%
{gooi (zoals in de tekst)}%
{gooide}%
{gooit}%
{gooien}

\vraag{22}{Het was een ridderkaar!}%
{ridderkaar}%
{ridderkaar (zoals in de tekst)}%
{ridderkaart}%
{rider kaart}%
{ridderskaart}

\vraag{25}{Volgend keer moet ik beter op mijn punten letten.}%
{moet}%
{moet (zoals in de tekst)}%
{moest}%
{moeten}%
{zal moeten}
\end{vragen}

\end{opgave}

\begin{oplossing}
\begin{enumerate}
\item B - bouwde is de correcte verleden tijd (niet bouwdte, de -de komt na
de stam bouw)
\item B - hadden is correct (meervoud: wij hadden)
\item B - gooide past bij de verleden tijd van het verhaal (niet plotseling
tegenwoordige tijd)
\item B - ridderkaart is de juiste spelling (dubbele a in kaart)
\item B - moest past bij het verleden tijdsperspectief van de verteller die
terugkijkt
\end{enumerate}
\end{oplossing}
