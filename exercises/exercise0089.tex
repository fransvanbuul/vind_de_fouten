\begin{opgave}
Familie Jansen maakt een autorit van Amsterdam naar Parijs voor hun 
zomervakantie. De afstand is 512 kilometer. Hun auto verbruikt 1 liter benzine 
per 14 kilometer. Benzine kost 1 euro en 75 cent per liter. Onderweg betalen 
ze 28 euro tol. Hoeveel kosten de brandstof en tol samen voor deze reis?
\end{opgave}

\begin{oplossing}
Benzineverbruik berekenen:
512 km : 14 km per liter = 36,57... liter

Afgerond naar boven: 37 liter (je kunt geen halve liter tanken)

Benzinekosten:
37 liter x 1,75 euro = 64,75 euro

Totale kosten:
64,75 + 28 = 92,75 euro

Antwoord: 92,75 euro
\end{oplossing}