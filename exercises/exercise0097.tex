\begin{opgave}
Er valt een meteoriet in de achtertuin van familie Peters! De meteoriet maakt 
verschillende (vierkante) kraters. De grootste krater heeft een doorsnede van 8 meter, de 
middelste krater heeft een doorsnee van 5 meter, en er zijn 6 kleine kraters 
met elk een doorsnede van 2 meter. Als het kost 35 euro per vierkante meter 
om de kraters te vullen, hoeveel kost het om alle kraters te repareren? 
(oppervlakte = doorsnede x doorsnede)
\end{opgave}

\begin{oplossing}
Oppervlaktes berekenen:
Grootste krater: 8 x 8 = 64 vierkante meter
Middelste krater: 5 x 5 = 25 vierkante meter
Kleine kraters: 6 x (2 x 2) = 6 x 4 = 24 vierkante meter

Totale oppervlakte:
64 + 25 + 24 = 113 vierkante meter

Reparatiekosten:
113 x 35 = 3955 euro

Antwoord: 3955 euro
\end{oplossing}