\begin{opgave}
De tennisclub moet nieuwe uitrusting kopen voor het jeugdteam. Er zijn 16 spelers in het team.
De kosten zijn:

\begin{itemize}
\item Tennisrackets: 85 euro per stuk
\item Tennisballen: 12 euro per koker (elke speler krijgt 2 kokers)
\item Sportshirts: 28 euro per stuk
\end{itemize}

De club heeft een budget van 2.500 euro. Hoeveel geld blijft er over nadat ze alles hebben
gekocht? Als er genoeg over is, kunnen ze dan nog polsbandjes kopen voor 6 euro per stuk?
\end{opgave}

\begin{oplossing}
Rackets: 16 x 85 = 1.360 euro
Ballen: 16 x 2 x 12 = 384 euro  
Shirts: 16 x 28 = 448 euro
Totaal: 1.360 + 384 + 448 = 2.192 euro
Over: 2.500 - 2.192 = 308 euro
Polsbandjes: 16 x 6 = 96 euro
308 - 96 = 212 euro over na polsbandjes
Antwoord: 212 euro over
\end{oplossing}