\begin{opgave}
Je speelt saxofoon in een jazzband en de band heeft verschillende optredens gepland.
De verdiensten per optreden zijn:

\begin{itemize}
\item Café optreden: \texteuro{85} per avond
\item Bruiloft: \texteuro{350} per evenement
\item Festival: \texteuro{180} per dag
\item Privéfeest: \texteuro{120} per avond
\end{itemize}

Deze maand heb je 4 café optredens, 2 bruiloften, 3 festivals en 5 privéfeesten
gespeeld. Van de totale inkomsten moet je 30\% afdragen aan je bandleider voor
organisatie. Hoeveel geld krijg jij?
\end{opgave}

\begin{oplossing}
Inkomsten:
Café optredens: 4 x \texteuro{85} = \texteuro{340}
Bruiloften: 2 x \texteuro{350} = \texteuro{700}
Festivals: 3 x \texteuro{180} = \texteuro{540}
Privéfeesten: 5 x \texteuro{120} = \texteuro{600}
Totaal: \texteuro{340} + \texteuro{700} + \texteuro{540} + \texteuro{600} = \texteuro{2.180}

Afdracht bandleider: 30\% van \texteuro{2.180} = \texteuro{654}
Jij krijgt: \texteuro{2.180} - \texteuro{654} = \texteuro{1.526}

Antwoord: \texteuro{1.526}
\end{oplossing}