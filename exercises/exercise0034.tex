\begin{opgave}
Link heeft een spaarplan voor nieuwe uitrusting. Hij verdeelt zijn dagelijkse
rupee-inkomsten in de verhouding 3:2:1 voor respectievelijk uitgaven, sparen
en donaties aan de Goddess Statue.

Deze week heeft hij de volgende dagelijkse inkomsten gehad:
Maandag: 180 rupees, Dinsdag: 240 rupees, Woensdag: 300 rupees,
Donderdag: 150 rupees, Vrijdag: 210 rupees.

Hoeveel rupees heeft Link deze week gespaard? Hoeveel rupees heeft hij
gedoneerd aan de Goddess Statue?
\end{opgave}

\begin{oplossing}
Totale inkomsten: 180 + 240 + 300 + 150 + 210 = 1.080 rupees

Verhouding 3:2:1 betekent totaal 3 + 2 + 1 = 6 delen

Per deel: 1.080 \textrm{\textdiv} 6 = 180 rupees per deel

Uitgaven: 3 x 180 = 540 rupees
Sparen: 2 x 180 = 360 rupees  
Donaties: 1 x 180 = 180 rupees

Controle: 540 + 360 + 180 = 1.080 rupees

Antwoord: 360 rupees gespaard en 180 rupees gedoneerd
\end{oplossing}