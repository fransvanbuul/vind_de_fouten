\begin{opgave}
Voor een saxofoon masterclass moet de organisatie een geschikte ruimte huren.
Ze bekijken verschillende opties:

\begin{itemize}
\item Kleine zaal: 8 meter x 6 meter, \texteuro{25} per vierkante meter per dag
\item Middelgrote zaal: 12 meter x 9 meter, \texteuro{22} per vierkante meter per dag
\item Grote zaal: 15 meter x 11 meter, \texteuro{20} per vierkante meter per dag
\item Concertzaal: 18 meter x 14 meter, \texteuro{30} per vierkante meter per dag
\end{itemize}

De masterclass duurt 3 dagen. Welke zaal kost het minst? Wat is het verschil
in totale kosten tussen de goedkoopste en duurste optie?
\end{opgave}

\begin{oplossing}
Oppervlaktes en kosten per dag:
Kleine zaal: 8 x 6 = 48 m², 48 x \texteuro{25} = \texteuro{1.200} per dag
Middelgrote zaal: 12 x 9 = 108 m², 108 x \texteuro{22} = \texteuro{2.376} per dag
Grote zaal: 15 x 11 = 165 m², 165 x \texteuro{20} = \texteuro{3.300} per dag
Concertzaal: 18 x 14 = 252 m², 252 x \texteuro{30} = \texteuro{7.560} per dag

Totale kosten voor 3 dagen:
Kleine zaal: 3 x \texteuro{1.200} = \texteuro{3.600}
Middelgrote zaal: 3 x \texteuro{2.376} = \texteuro{7.128}
Grote zaal: 3 x \texteuro{3.300} = \texteuro{9.900}
Concertzaal: 3 x \texteuro{7.560} = \texteuro{22.680}

Goedkoopste: Kleine zaal (\texteuro{3.600})
Duurste: Concertzaal (\texteuro{22.680})
Verschil: \texteuro{22.680} - \texteuro{3.600} = \texteuro{19.080}

Antwoord: Kleine zaal goedkoopst, verschil \texteuro{19.080}
\end{oplossing}