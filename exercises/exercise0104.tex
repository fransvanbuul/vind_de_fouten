\begin{opgave}

\begin{tekstmetfouten}
Vorige week heb ik het laatste boek van Harry Potter uitgelezen.\\
Het verhaal gaat over een jongen die naar een toverschool gaat\\
die Zweinstein heet. Op die school leert hij allerlei magische\\
spreuken en bezweringen. Mijn favoriete personage is Hermelien\\
omdat zij zo slim is en altijd de beste van de klas. In het\\
verhaal moet Harry Potter vechten tegen de boze tovenaar\\
Voldemort. Dat karakter vind ik heel eng omdat hij geen neus\\
heeft en er zo raar uitziet. In het begin van het verhaal wonen\\
Harry nog bij zijn gemene oom en tante. Ze behandelen hem heel\\
slecht en laten hem in een kast onder de trap slapen. Dat vond\\
ik zielig. Gelukkig komt er op een dag een reus die Hagrid heet.\\
Hij vertelt Harry dat hij een tovenaar is en dat hij naar\\
Zweinstein mag. Die school ligt ergens in Schotland verstopt.\\
Gewone mensen kunnen de school niet zien want er ligt een\\
betovering op. In het boek zijn er vier afdelingen op school.\\
Harry komt in Griffoendor, samen met zijn vrienden Ron en\\
Hermelien. De leraren op school zijn soms streng maar meestal\\
aardig. Professor Sneep is de enige die Harry niet mag. Hij geeft\\
toverdranken en is altijd oneerlijk tegen Harry. Mijn broer heeft\\
alle films ook gezien en die vindt het verhaal net zo spannend\\
als ik. Volgende maand ga ik naar de Harry Potter studio in\\
Londen. Daar kan je alle decors en kostuums uit de films zien.\\
Ik verheug me er heel erg op. Misschien koop ik daar ook een\\
toverstaf als souvenir.
\end{tekstmetfouten}

\begin{vragen}
\vraag{7}{Dat karakter vind ik heel eng ... raar uitziet.}%
{vind}%
{vind (zoals in de tekst)}%
{vinden}%
{vindt}%
{vond}

\vraag{9}{In het begin ... wonen Harry nog bij ... tante.}%
{wonen Harry}%
{wonen Harry (zoals in de tekst)}%
{woont Harry}%
{woonde Harry}%
{wonen Harrie}

\vraag{13}{Die school ligt ... Schotland verstopt.}%
{Die school}%
{Die school (zoals in de tekst)}%
{Deze school}%
{Dat school}%
{Dit school}

\vraag{14}{Gewone mensen ... want er ligt een betovering op.}%
{want er}%
{want er (zoals in de tekst)}%
{want, er}%
{want Er}%
{want: er}

\vraag{22}{Daar kan je alle decors en kostuums ... films zien.}%
{kan je}%
{kan je (zoals in de tekst)}%
{kun je}%
{kunt je}%
{kan jij}

\end{vragen}

\end{opgave}

\begin{oplossing}
\begin{enumerate}
\item C - vindt is correct (3e persoon enkelvoud tegenwoordige tijd: dat
vind-t)
\item C - woont Harry is correct (enkelvoud, geen meervoud)
\item A - Die school is correct (de-woord, ver weg, dus die)
\item B - want, er is correct (komma voor nevenschikkend voegwoord want)
\item B - kun je is de juiste spelling (kunnen wordt kun in deze vorm, niet
kan)
\end{enumerate}
\end{oplossing}
