\begin{opgave}
De scoutinggroep gaat op zomerkamp. Ze hebben 420 euro gespaard voor kampuitrusting. Ze
willen het geld gelijk verdelen over drie groepen. Elke groep gaat de volgende spullen kopen:

\begin{itemize}
\item Een grote tent kost 95 euro
\item Slaapzakken kosten 35 euro per stuk (elke groep heeft 4 leden)
\item Een kampkookset kost 25 euro
\end{itemize}

Hoeveel euro krijgt elke groep? Kunnen ze alle benodigde spullen kopen? Zo ja, hoeveel
houden ze dan per groep over?
\end{opgave}

\begin{oplossing}
Geld per groep: 420 : 3 = 140 euro per groep.
Kosten per groep: 95 + (4 x 35) + 25 = 95 + 140 + 25 = 260 euro.
Dit is meer dan 140 euro, dus ze hebben niet genoeg geld.
Antwoord: 140 euro per groep, maar niet genoeg voor alle spullen
\end{oplossing}