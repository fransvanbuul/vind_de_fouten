\begin{opgave}
Je start een kookblog en wilt professionele kookboeken en keukengerei kopen.
Je budget is \texteuro{450}. De prijzen zijn:

\begin{itemize}
\item Kookboeken: \texteuro{25} per stuk, je wilt er 8
\item Pannenset: \texteuro{89} voor de complete set
\item Messenblok: \texteuro{67}
\item Keukenmachine: \texteuro{145}
\end{itemize}

Je vindt een aanbieding: bij aankoop van 6 of meer kookboeken krijg je 20\%
korting op de totale boekenprijs. Kun je alles kopen? Hoeveel geld heb je
over of tekort?
\end{opgave}

\begin{oplossing}
Kookboeken met korting:
8 boeken x \texteuro{25} = \texteuro{200}
Korting (20\%): \texteuro{200} x 0,20 = \texteuro{40}
Kookboeken na korting: \texteuro{200} - \texteuro{40} = \texteuro{160}

Overige kosten:
Pannenset: \texteuro{89}
Messenblok: \texteuro{67}
Keukenmachine: \texteuro{145}

Totaal: \texteuro{160} + \texteuro{89} + \texteuro{67} + \texteuro{145} = \texteuro{461}
Budget: \texteuro{450}
Te kort: \texteuro{461} - \texteuro{450} = \texteuro{11}

Antwoord: \texteuro{11} te kort
\end{oplossing}