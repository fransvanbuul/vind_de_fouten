\begin{opgave}
Meneer Bakker rijdt elke week van Utrecht naar Rotterdam voor zijn werk. De 
afstand is 65 kilometer per richting. Hij maakt deze rit 4 keer per week 
(heen en terug). Zijn auto verbruikt 1 liter benzine per 16 kilometer. 
Benzine kost 1 euro en 82 cent per liter. Hoeveel geeft hij per week uit aan 
benzine voor deze ritten?
\end{opgave}

\begin{oplossing}
Afstand per week berekenen:
Per rit heen en terug: 65 x 2 = 130 km
Per week (4 keer): 4 x 130 = 520 km

Benzineverbruik:
520 km : 16 km per liter = 32,5 liter

Benzinekosten:
32,5 liter x 1,82 euro = 59,15 euro

Antwoord: 59,15 euro
\end{oplossing}