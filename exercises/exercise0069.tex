\begin{opgave}
David start een Scrabble club op school. Hij koopt het volgende:
\begin{itemize}
\item 5 Scrabble sets voor 22 euro per set
\item 3 woordenboeken voor 15 euro per stuk
\item 8 scoreborden voor 6 euro per stuk  
\item 2 timers voor 18 euro per stuk
\end{itemize}
De school geeft hem 75 euro subsidie. Hoeveel moet David zelf bijbetalen?
\end{opgave}

\begin{oplossing}
Kosten per item berekenen:
Scrabble sets: 5 x 22 = 110 euro
Woordenboeken: 3 x 15 = 45 euro
Scoreborden: 8 x 6 = 48 euro
Timers: 2 x 18 = 36 euro

Totale kosten:
110 + 45 + 48 + 36 = 239 euro

Eigen bijdrage:
239 - 75 = 164 euro

Antwoord: 164 euro
\end{oplossing}