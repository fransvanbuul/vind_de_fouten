\begin{opgave}
De tennisclub organiseert een jeugdtoernooi. Het inschrijfgeld verschilt per leeftijd:

\begin{itemize}
\item 8-10 jaar: \texteuro{12} per persoon
\item 11-13 jaar: \texteuro{15} per persoon
\item 14-16 jaar: \texteuro{18} per persoon
\end{itemize}

Er hebben zich ingeschreven: 25 kinderen van 8-10 jaar, 32 kinderen van 11-13 jaar
en 18 kinderen van 14-16 jaar. De club moet ook \texteuro{150} betalen voor het
huren van extra banen en \texteuro{85} voor prijzen. Hoeveel geld houdt de club
over na alle kosten?
\end{opgave}

\begin{oplossing}
Inkomsten:
8-10 jaar: 25 x \texteuro{12} = \texteuro{300}
11-13 jaar: 32 x \texteuro{15} = \texteuro{480}
14-16 jaar: 18 x \texteuro{18} = \texteuro{324}
Totale inkomsten: \texteuro{300} + \texteuro{480} + \texteuro{324} = \texteuro{1.104}

Kosten:
Banenhuur: \texteuro{150}
Prijzen: \texteuro{85}
Totale kosten: \texteuro{150} + \texteuro{85} = \texteuro{235}

Over: \texteuro{1.104} - \texteuro{235} = \texteuro{869}

Antwoord: \texteuro{869}
\end{oplossing}